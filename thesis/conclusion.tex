\chapter{Conclusions}

\section{Findings}

	\subsection{Hypotheses Are Not Refuted}

In Chapter \ref{intro-hypo-one} on page \pageref{intro-hypo-one} we formulated two hypotheses, which we can not refute after the evaluation.

\hypothesis{$H_1$}{\hypoOne}

The evaluation revealed a linear increase in runtime in the interval of $[1,1000]$ configuration settings.
We can not refute $H_1$ with our benchmark results.

\hypothesis{$H_2$}{\hypoTwo}

The evaluation shows a measureable but constant memory overhead in the interval of $[1,200]$ configuration settings.
We can not refute $H_2$ with our benchmark results.

	\subsection{Answers to the Research Questions}

In Section \ref{researchq} on page \pageref{researchq} we defined our research questions.
After what we learned during the evaluation we are able to give some answers.

\RQ{$RQ_1$}{\rqOne}

When comparing the following providers of cryptographic functions:
\begin{enumerate}
\item libgcrypt,
\item OpenSSL, and
\item Botan,
\end{enumerate}

libgcrypt has to lowest runtime impact in the interval of $[1,1000]$ configuration settings.

The lowest memory impact could not be found in the evaluation, because libgcrypt does not use a custom allocator.
When comparing OpenSSL against Botan with $n \in \{10,100,200\}$ configuration settings, Botan allocates 6.0 kiB less heap memory.

\RQ{$RQ_2$}{\rqTwo}

Let $i$ be the number of configuration settings in a benchmark result.
Let $x_i$ denote the runtime without the \crypto.
Furthermore let $y_i$ denote the runtime with a plugin variant of the \crypto.
Then the overhead $h_i$ is given as: $h_i=y_i - x_i$.

Let $n$ be the number of measurements.
Then the average runtime overhead $h$ is given as: $h=(1/n) \sum_{i=1}^{n}(y_i - x_i)$.

The average runtime overhead factor $f$ is given as: $f=(1/n) \sum_{i=1}^{n}(y_i/x_i)$.

Table \ref{concl-time-crypto} on page \pageref{concl-time-crypto} shows the overhead $h$ for every plugin variant of the \crypto~ in the interval $[1,1000]$ configuration settings.

\begin{table}[h]
\centering
\caption{Average overhead of the \crypto}
\label{concl-time-crypto}
\begin{tabular}{l|rr|rr}
          & \multicolumn{2}{c|}{Decryption} & \multicolumn{2}{c}{Encryption} \\ \cline{2-5}
          & $h$ (s)      & $f$           & $h$ (s)       & $f$          \\ \hline
libgcrypt & $1.168$ s    & $1837.868$    & $1.171$ s     & $192.912$    \\
OpenSSL   & $1.805$ s    & $2835.492$    & $1.808$ s     & $297.236$    \\
Botan     & $4.125$ s    & $6459.077$    & $4.124$ s     & $675.795$
\end{tabular}
\end{table}

The question for the memory overhead can only be partially answered, due to the lack of a custom allocating function in libgcrypt.
The results are given in Table \ref{eval-mem-res} on page \pageref{eval-mem-res}.

\RQ{$RQ_3$}{\rqThree}

Table \ref{concl-time-fcrypt} on page \pageref{concl-time-fcrypt} shows the overhead $h$ for the \fcrypt~ in the interval $[1,1000]$ configuration settings.

\begin{table}[h]
\centering
\caption{Average overhead of the \fcrypt}
\label{concl-time-fcrypt}
\begin{tabular}{l|rr|rr}
          & \multicolumn{2}{c|}{Decryption} & \multicolumn{2}{c}{Encryption} \\ \cline{2-5}
          & $h$ (s)      & $f$           & $h$ (s)       & $f$          \\ \hline
fcrypt    & $0.005$ s    & $25.609$      & $0.007$ s     & $2.791$     
\end{tabular}
\end{table}


The question for the memory overhad can not be answered after the evaluation.
Further benchmarks are required to gain more insights.

\section{Discussion}

In this section we want to discuss drawbacks of the methodology we used in our thesis.

	\subsection{Isolated Performance Test}

The benchmarks are designed as isolated (artificial) tests.
The test results do not neccessarily correlate with ``real-world'' use cases.
For example: we did not run benchmarks with configuration settings, that contain values that are not to be encrypted (mixed configuration settings).

Another problem is that the interval boundries of the benchmarks are limitied.
However, we are able to draw conclusions about the runtime overhead tendency after the evaluation.

	\subsection{Variety of Cryptographic Schemas}

We evaluated only two cryptographic schemas:
\begin{enumerate}
\item the \crypto, and
\item the \fcrypt.
\end{enumerate}

Altough the plugins are carefully designed, it is possible that the runtime overhead is distorted by the plugin design.

\section{Résumé}

\todo{TBD}
