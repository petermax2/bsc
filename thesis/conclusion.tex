\chapter{Conclusions}

\section{Findings}

	\subsection{Hypotheses Are Not Refuted}

In Chapter \ref{intro-hypo-one} on page \pageref{intro-hypo-one} we formulated two hypotheses, which we can not refute after the evaluation.

\hypothesis{$H_1$}{\hypoOne}

The evaluation revealed a linear increase in runtime in the interval of $[1,1000]$ configuration settings.
We can not refute $H_1$ with our benchmark results.

\hypothesis{$H_2$}{\hypoTwo}

The evaluation shows a measureable but constant memory overhead in the interval of $[1,200]$ configuration settings.
We can not refute $H_2$ with our benchmark results.

	\subsection{Answers to the Research Questions}

In Section \ref{researchq} on page \pageref{researchq} we defined our research questions.
After what we learned during the evaluation we are able to give some answers.

\RQ{$RQ_1$}{\rqOne}

When comparing the following providers of cryptographic functions:
\begin{enumerate}
\item libgcrypt,
\item OpenSSL, and
\item Botan,
\end{enumerate}

libgcrypt has to lowest runtime impact in the interval of $[1,1000]$ configuration settings.

The lowest memory impact could not be found in the evaluation, because libgcrypt does not use a custom allocator.
When comparing OpenSSL against Botan with $n \in \{10,100,200\}$ configuration settings, Botan allocates 6.0 kiB less heap memory.

\RQ{$RQ_2$}{\rqTwo}

Let $i$ be the number of configuration settings in a benchmark result.
Let $x_i$ denote the runtime without the \crypto.
Furthermore let $y_i$ denote the runtime with a plugin variant of the \crypto.
Then the overhead $h_i$ is given as: $h_i=y_i - x_i$.

Let $n$ be the number of measurements.
Then the average runtime overhead $h$ is given as: $h=(1/n) \sum_{i=1}^{n}(y_i - x_i)$.

The average runtime overhead factor $f$ is given as: $f=(1/n) \sum_{i=1}^{n}(y_i/x_i)$.

Table \ref{concl-time-crypto} on page \pageref{concl-time-crypto} shows the overhead $h$ for every plugin variant of the \crypto~ in the interval $[1,1000]$ configuration settings.

\begin{table}[h]
\centering
\caption{Average overhead of the \crypto}
\label{concl-time-crypto}
\begin{tabular}{l|rr|rr}
          & \multicolumn{2}{c|}{Decryption} & \multicolumn{2}{c}{Encryption} \\ \cline{2-5}
          & $h$ (s)      & $f$           & $h$ (s)       & $f$          \\ \hline
libgcrypt & $1.168$ s    & $1837.868$    & $1.171$ s     & $192.912$    \\
OpenSSL   & $1.805$ s    & $2835.492$    & $1.808$ s     & $297.236$    \\
Botan     & $4.125$ s    & $6459.077$    & $4.124$ s     & $675.795$
\end{tabular}
\end{table}

The question for the memory overhead can only be partially answered, due to the lack of a custom allocating function in libgcrypt.
The results are given in Table \ref{eval-mem-res} on page \pageref{eval-mem-res}.

\RQ{$RQ_3$}{\rqThree}

Table \ref{concl-time-fcrypt} on page \pageref{concl-time-fcrypt} shows the overhead $h$ for the \fcrypt ~in the interval $[1,1000]$ configuration settings.

\begin{table}[h]
\centering
\caption{Average overhead of the \fcrypt}
\label{concl-time-fcrypt}
\begin{tabular}{l|rr|rr}
          & \multicolumn{2}{c|}{Decryption} & \multicolumn{2}{c}{Encryption} \\ \cline{2-5}
          & $h$ (s)      & $f$           & $h$ (s)       & $f$          \\ \hline
fcrypt    & $0.005$ s    & $25.609$      & $0.007$ s     & $2.791$     
\end{tabular}
\end{table}

The question for the memory overhad can not be answered after the evaluation.
Further benchmarks are required to gain more insights.

\section{Discussion}

In this section we:
\begin{enumerate}
\item discuss drawbacks of the methodology we used in our thesis, and
\item interpret the meaning of the evaluation results.
\end{enumerate}

	\subsection{Isolated Performance Test}

The benchmarks are designed as isolated (artificial) tests.
The test results do not neccessarily correlate with ``real-world'' use cases.
For example: we did not run benchmarks with configuration settings, that contain values that are not to be encrypted (mixed configuration settings).

	\subsection{Variety of Cryptographic Schemas}

We evaluated only two cryptographic schemas:
\begin{enumerate}
\item the \crypto, and
\item the \fcrypt.
\end{enumerate}

Altough the plugins are carefully designed, it is possible that the runtime overhead is distorted by the plugin design.
Other reference environments than Elektra should be used to repeat the benchmarks.

	\subsection{Limited Hardware}

The benchmarks were executed on a single system.
Especially older processors without built-in AES support, will probably show a higher runtime overhead.
The benchmarks should therefore be repeated with different hardware configurations.

	\subsection{Interpretation of the Results}

When encrypting and decrypting configuration settings and configuration files, the expected runtime overhead increases linear with the number of configuration values.
The explanation for this behavior is the process of how the plugins operate.
The \crypto ~iterates over the whole configuration setting, thus the linear increase in runtime.
The \fcrypt ~encrypts and decrypts whole files.
The bigger the size of the configuration setting, the bigger the size of the resulting configuration file.
Both plugins have a tendency towards linear runtime overhead.

The increase of memory consumption is constant, due to the design of the plugins.
The \crypto ~encrypts and decrypts single configuration values only, so only one configuration value is being processed at a time.
Since every configuration value in our benchmark has the same size, the memory allocations stay the same, even when the size of the configuration setting increases.

\chapter{Résumé}

After introducing the problem of missing cryptographic methods for software configuration, we presented Elektra.
We learned about Elektra's configuration database and the plugin system.
We explained how the \crypto, the \fcrypt ~and some helper plugins have been designed and developed to study our research questions and solve the problem of encrypting and decrypting configuration settings.

Our evaluation revealed a linear increase in runtime overhead when cryptography is applied to configuration settings in our benchmark setup.
The memory analysis showed that the increase in memory overhead is constant in our benchmark setup for different configuration setting sizes.
Therefore we were not able to refute the hypotheses, that we stated in the introduction.

If we compare the evaluated cryptographic schemas, then the \fcrypt~ is faster than the \crypto.
In comparison to Botan and OpenSSL, libgcrypt shows the best runtime performance in our benchmark setup.
