\chapter{Conclusions}

\section{Findings}

	\subsection{Hypotheses Are Not Refuted}

In Chapter \ref{intro-hypo-one} on page \pageref{intro-hypo-one} we formulated two hypotheses, which we can not refute after the evaluation.

\hypothesis{$H_1$}{\hypoOne}

The evaluation revealed a linear increase in runtime in the interval of $[1,1000]$ configuration settings.
We can not refute $H_1$ with our benchmark results.

\hypothesis{$H_2$}{\hypoTwo}

The evaluation shows a measureable but constant memory overhead in the interval of $[1,200]$ configuration settings.
We can not refute $H_2$ with our benchmark results.

	\subsection{Answering the Research Questions}

In Section \ref{researchq} on page \pageref{researchq} we defined our research questions.
After what we learned during the evaluation we are able to give some answers.

\RQ{$RQ_1$}{\rqOne}

When comparing the following providers of cryptographic functions:
\begin{enumerate}
\item libgcrypt,
\item OpenSSL, and
\item Botan,
\end{enumerate}

libgcrypt has to lowest runtime impact in the interval of $[1,1000]$ configuration settings.

\RQ{$RQ_2$}{\rqTwo}

\todo{tbd}

\RQ{$RQ_3$}{\rqThree}

\todo{tbd}

\section{Discussion}

In this section we want to discuss drawbacks of the methodology we used in our thesis.

	\subsection{Isolated Performance Test}

The benchmarks are designed as isolated (artificial) tests.
The test results do not neccessarily correlate with ``real-world'' use cases.
For example: we did not run benchmarks with configuration settings, that contain values that are not encrypted.

Another problem is that the interval boundries of the benchmarks are limitied.
However, we are able to draw conclusions about the runtime overhead tendency after the evaluation.

	\subsection{Number of Cryptograhic Schemas}

We evaluated only two cryptographic schemas, which are the \elektra~ plugins:
\begin{enumerate}
\item \crypto, and
\item \fcrypt.
\end{enumerate}

Altough the plugins were carefully designed, it is possible that the runtime overhead is distorted by the plugin design.

	\subsection{Other criteria for choosing providers of cryptographic functions}

Performance is not the only factor to be taken into account when choosing a provider of cryptographic functions.
Robustness against attacks (for example: side channel attacks) and correctness of the code are two important dimensions, that should also be considered.

Also the usability and the user acceptance play an important role in the decision process.
We did not cover any of those aspects due to the limited context of the thesis.

\section{Résumé}

\todo{TBD}
