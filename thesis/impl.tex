\chapter{Implementation}

This chapter describes the changes and enhancements that were made to Elektra in order to answer the research questions.

First we talk about Elektra's plugin API in general and how we used it to introduce cryptographic operations to Elektra.
Then the concept of the key management is presented.
At last some interesting implementation details about the crypto libraries used are given.

\section{Elektra Plugins}

Elektra abstracts configuration parameters in a hierarchical key-value database.
A KeySet holds zero or more Keys.
The (Elektra) Key holds the configuration parameter either as a string or as a binary value.
The core of Elektra is kept small, meaning that it provides mainly the database abstraction as well as a plugin system.
All the configuration access operations (mainly file reads and writes but there are more exotic constructs as well) are performed by plugins.
The plugins should fulfill exactly one purpose, keeping to the UNIX philosophy.
To give an example: one plugin may write to /etc/host and another one may encode binary values using the Base64 encoding scheme.
Multiple plugins may be combined into a backend.

Backends use the concept of mounting like UNIX-like file systems.
Every backend has its own configuration itself, which provides influence on the runtime behavior of the plugins within the backend.

A plugin can export different methods in order to fulfill their purpose:

    \subsection{checkconf}

    \subsection{open}

    \subsection{close}

    \subsection{set}

    \subsection{get}



\section{Key Management}

\section{Details About The Crypto Libraries}
