\chapter{Experimental Evaluation}

\section{Methodology}

\todo{come up with some intro...}
\begin{itemize}
  \item method: benchmark
  \item why?
  \item how? - general description
\end{itemize}

see valgrind manual \cite{valgrind}.

\section{System Setup}
  \subsection{Hardware Setup}

The processor to run the benchmarks is an Intel\textregistered~ Core\texttrademark~ i7-4771 CPU clocked at 3.50 GHz with 4 cores and hyperthreading support.
It has 15825 MB of DDR3 RAM available clocked at 1333 MHz.
The amount of total memory available was measured with the \texttt{free} program

The operating system (root partition) is located on a ``Samsung SSD 840'' solid state drive.
It uses the XFS filesystem inside an encrypted LUKS\footnote{Linux Unified Key Setup} container within a LVM\footnote{Logival Volume Manager} partition.

The home directory is located on a software RAID 1\footnote{Redundant Array Of Inexpensive Disks} array, containing of two ``ST1000DM003-1SB1'' hard disks.
It also uses the XFS filesystem inside an encrypted LUKS container within a LVM partiton.

See the output of the \texttt{lsblk} program for the exact hard drive arrangement.

\begin{lstlisting}[caption={Disk arrangement on the benchmark machine}]
NAME                          TYPE  MOUNTPOINT
sda                           disk
|-sda1                        part  /boot/efi
|-sda2                        part  /boot
+-sda3                        part
  +-luks-6bcb2406             crypt
    |-centos_peterspc-root    lvm   /
    +-centos_peterspc-swap    lvm   [SWAP]
sdb                           disk
+-sdb1                        part
  +-md127                     raid1
    +-home                    crypt /home
sdc                           disk
+-sdc1                        part
  +-md127                     raid1
    +-home                    crypt /home
\end{lstlisting}

The \texttt{hdparm} program gives an idea of how much throughput this disk setup can actually handle:

\begin{lstlisting}[caption={Disk performance on the benchmark machine}]
/dev/sda:
 Timing cached reads:   30192 MB in  2.00 seconds = 15133.81 MB/sec
 Timing buffered disk reads: 1548 MB in  3.00 seconds = 515.86 MB/sec

/dev/md127:
 Timing cached reads:   30608 MB in  1.99 seconds = 15343.38 MB/sec
 Timing buffered disk reads: 628 MB in  3.01 seconds = 208.78 MB/sec
\end{lstlisting}

The test configuration for the benchmarks are stored in the home directory (RAID 1 array).

  \subsection{Software Setup}

The software stack used to build and run the benchmarks is entirely defined by CentOS Linux release 7.2.1511 (Core).
We highlight the most important program versions nevertheless:

\begin{itemize}
  \item GNU Make version 3.82
  \item cmake version 2.8.11
  \item gcc version 4.8.5 20150623 (Red Hat 4.8.5-4)
  \item glibc version 2.17-106.el7\textunderscore 2.8
  \item valgrind version 3.10.0
\end{itemize}

The Linux kernel version used for running the tests is 3.10.0-327.36.3.el7.x86\textunderscore 64.

\section{Benchmark 1 -- TBD}

\section{Interpretation Of The Results}
