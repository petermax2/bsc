\chapter{Experimental Evaluation}

\section{Methodology}

Runtime and memory usage evaluation are the focus of this chapter.
Both parameters are quantified by benchmarks.

The runtime of a benchmark will be measured using the system time, which is returned by the \texttt{gettimeofday ()} system function.

The memory usage is examined using the ``Massif'' tool which is part of the \texttt{valgrind} suite.\cite{valgrind}

Before we dive into the details about the benchmarks, the system setup on which they will be performed is described.

\section{System Setup}
  \subsection{Hardware Setup}

The processor to run the benchmarks is an Intel\textregistered~ Core\texttrademark~ i7-4771 CPU clocked at 3.50 GHz with 4 cores and hyperthreading support.
It has 15825 MB of DDR3 RAM available clocked at 1333 MHz.
The amount of total memory available was measured with the \texttt{free} program

The operating system (root partition) is located on a ``Samsung SSD 840'' solid state drive.
It uses the XFS filesystem inside an encrypted LUKS\footnote{Linux Unified Key Setup} container within a LVM\footnote{Logival Volume Manager} partition.

The home directory is located on a software RAID 1\footnote{Redundant Array Of Inexpensive Disks} array, containing of two ``ST1000DM003-1SB1'' hard disks.
It also uses the XFS filesystem inside an encrypted LUKS container.

See the output of the \texttt{lsblk} program for the exact hard drive arrangement.

\begin{lstlisting}[caption={Disk arrangement on the benchmark machine}]
NAME                          TYPE  MOUNTPOINT
sda                           disk
|-sda1                        part  /boot/efi
|-sda2                        part  /boot
+-sda3                        part
  +-luks-6bcb2406             crypt
    |-centos_peterspc-root    lvm   /
    +-centos_peterspc-swap    lvm   [SWAP]
sdb                           disk
+-sdb1                        part
  +-md127                     raid1
    +-home                    crypt /home
sdc                           disk
+-sdc1                        part
  +-md127                     raid1
    +-home                    crypt /home
\end{lstlisting}

The \texttt{hdparm} program gives an idea of how much throughput this disk setup can actually handle:

\begin{lstlisting}[caption={Disk performance on the benchmark machine}]
/dev/sda:
 Timing cached reads:   30192 MB in  2.00 seconds = 15133.81 MB/sec
 Timing buffered disk reads: 1548 MB in  3.00 seconds = 515.86 MB/sec

/dev/md127:
 Timing cached reads:   30608 MB in  1.99 seconds = 15343.38 MB/sec
 Timing buffered disk reads: 628 MB in  3.01 seconds = 208.78 MB/sec
\end{lstlisting}

The test configuration for the benchmarks are stored in the home directory (RAID 1 array).

  \subsection{Software Setup}

The software stack used to build and run the benchmarks is defined by CentOS Linux release 7.2.1511 (Core) with the exception of Elektra.
We highlight the most important standard program versions:

\begin{itemize}
  \item GNU Make version 3.82
  \item cmake version 2.8.11
  \item gcc version 4.8.5 20150623 (Red Hat 4.8.5-4)
  \item glibc version 2.17-106.el7\textunderscore 2.8
  \item valgrind version 3.10.0
  \item Botan version 1.10.13
  \item OpenSSL version 1.0.1e
  \item libgcrypt version 1.5.3
  \item libgpg-error version 1.12
  \item GnuPG version 2.0.22
\end{itemize}

The Linux kernel version used for running the tests is 3.10.0-327.36.3.el7.x86\textunderscore 64.

The modified Elektra source code is hosted on Github\footnote{\url{https://github.com/petermax2/libelektra}} in a separate branch called ``benchmarks''.

\section{Benchmark 1 -- TBD}

\section{Interpretation Of The Results}
