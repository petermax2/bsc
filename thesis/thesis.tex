% Copyright (C) 2014-2015 by Thomas Auzinger <thomas@auzinger.name>

\documentclass[draft,final]{vutinfth} % Remove option 'final' to obtain debug information.

% Load packages to allow in- and output of non-ASCII characters.
\usepackage{lmodern}        % Use an extension of the original Computer Modern font to minimize the use of bitmapped letters.
\usepackage[T1]{fontenc}    % Determines font encoding of the output. Font packages have to be included before this line.
\usepackage[utf8]{inputenc} % Determines encoding of the input. All input files have to use UTF8 encoding.

% Extended LaTeX functionality is enables by including packages with \usepackage{...}.
\usepackage{fixltx2e}   % Provides fixes for several errors in LaTeX2e.
\usepackage{amsmath}    % Extended typesetting of mathematical expression.
\usepackage{amssymb}    % Provides a multitude of mathematical symbols.
\usepackage{mathtools}  % Further extensions of mathematical typesetting.
\usepackage{microtype}  % Small-scale typographic enhancements.
\usepackage{enumitem}   % User control over the layout of lists (itemize, enumerate, description).
\usepackage{multirow}   % Allows table elements to span several rows.
\usepackage{booktabs}   % Improves the typesettings of tables.
\usepackage{subcaption} % Allows the use of subfigures and enables their referencing.
\usepackage[ruled,linesnumbered,algochapter]{algorithm2e} % Enables the writing of pseudo code.
\usepackage[usenames,dvipsnames,table]{xcolor} % Allows the definition and use of colors. This package has to be included before tikz.
\usepackage{nag}       % Issues warnings when best practices in writing LaTeX documents are violated.
\usepackage{hyperref}  % Enables cross linking in the electronic document version. This package has to be included second to last.
\usepackage[acronym,toc]{glossaries} % Enables the generation of glossaries and lists fo acronyms. This package has to be included last.

% Define convenience functions to use the author name and the thesis title in the PDF document properties.
\newcommand{\authorname}{Peter Nirschl} % The author name without titles.
\newcommand{\thesistitle}{Cryptographic Methods For Elektra} % The title of the thesis. The English version should be used, if it exists.

% Set PDF document properties
\hypersetup{
    pdfpagelayout   = TwoPageRight,           % How the document is shown in PDF viewers (optional).
    linkbordercolor = {Melon},                % The color of the borders of boxes around crosslinks (optional).
    pdfauthor       = {\authorname},          % The author's name in the document properties (optional).
    pdftitle        = {\thesistitle},         % The document's title in the document properties (optional).
    pdfsubject      = {},              % The document's subject in the document properties (optional).
    pdfkeywords     = {cryptography,elektra,performance} % The document's keywords in the document properties (optional).
}

\setsecnumdepth{subsection} % Enumerate subsections.

\nonzeroparskip             % Create space between paragraphs (optional).
\setlength{\parindent}{0pt} % Remove paragraph identation (optional).

\makeindex      % Use an optional index.
\makeglossaries % Use an optional glossary.
%\glstocfalse   % Remove the glossaries from the table of contents.

% Set persons with 4 arguments:
%  {title before name}{name}{title after name}{gender}
%  where both titles are optional (i.e. can be given as empty brackets {}).
\setauthor{}{\authorname}{}{female}
\setadvisor{DI}{Markus Raab}{}{male}

% For bachelor and master theses:
%\setfirstassistant{Pretitle}{Forename Surname}{Posttitle}{male}
%\setsecondassistant{Pretitle}{Forename Surname}{Posttitle}{male}
%\setthirdassistant{Pretitle}{Forename Surname}{Posttitle}{male}

% For dissertations:
%\setfirstreviewer{Pretitle}{Forename Surname}{Posttitle}{male}
%\setsecondreviewer{Pretitle}{Forename Surname}{Posttitle}{male}

% For dissertations at the PhD School:
%\setsecondadvisor{Pretitle}{Forename Surname}{Posttitle}{male}

% Required data.
\setaddress{Jochen-Rindt-Straße 10 Haus 24, 1230 Wien}
\setregnumber{1025647}
\setdate{01}{01}{2016}
\settitle{\thesistitle}{Cryptographic Methods For Elektra} % Sets English and German version of the title (both can be English or German).
\setsubtitle{}{} % Sets English and German version of the subtitle (both can be English or German).

% Select the thesis type: bachelor / master / doctor / phd-school.
% Bachelor:
\setthesis{bachelor}
%
% Master:
%\setthesis{master}
%\setmasterdegree{dipl.} % dipl. / rer.nat. / rer.soc.oec. / master
%
% Doctor:
%\setthesis{doctor}
%\setdoctordegree{rer.soc.oec.}% rer.nat. / techn. / rer.soc.oec.
%
% Doctor at the PhD School
%\setthesis{phd-school} % Deactivate non-English title pages (see below)

% For bachelor and master:
\setcurriculum{Software and Information Engineering}{Software and Information Engineering} % Sets the English and German name of the curriculum.

% For dissertations at the PhD School:
%\setfirstreviewerdata{Affiliation, Country}
%\setsecondreviewerdata{Affiliation, Country}

% Define convenience macros.
\newcommand{\todo}[1]{{\color{red}\textbf{TODO: {#1}}}} % Comment for the final version, to raise errors.


\begin{document}

\frontmatter % Switches to roman numbering.
% The structure of the thesis has to conform to
%  http://www.informatik.tuwien.ac.at/dekanat

\addtitlepage{naustrian} % German title page (not for dissertations at the PhD School).
\addtitlepage{english} % English title page.
\addstatementpage

\begin{danksagung*}
\todo{Ihr Text hier.}
\end{danksagung*}

\begin{acknowledgements*}
\todo{Enter your text here.}
\end{acknowledgements*}

\begin{kurzfassung}
\todo{Ihr Text hier.}
\end{kurzfassung}

\begin{abstract}
\todo{Enter your text here.}
\end{abstract}

% Select the language of the thesis, e.g., english or naustrian.
\selectlanguage{english}

% Add a table of contents (toc).
\tableofcontents % Starred version, i.e., \tableofcontents*, removes the self-entry.

% Switch to arabic numbering and start the enumeration of chapters in the table of content.
\mainmatter

\chapter{Introduction}
\label{intro}

Storing and providing login credentials is a common problem in software development.
Login credentials in this thesis refer to information that grants access to a system, for example:

\begin{enumerate}
\item passwords, and
\item access tokens (like OAuth tokens).
\end{enumerate}

Login credentials are seen as part of configuration settings of applications.

Applications prefereably store login credentials to related systems in configuration files.
A typical example is a business application that connects to a database system.
The login credentials are often saved as plain text, leaving them vulnerable to attack.
We want to introduce the problem by giving two concrete examples:

\begin{enumerate}
\item WordPress, and
\item Hibernate.
\end{enumerate}

WordPress is a typical web application with a backend connecting to a database.
WordPress reads the login credentials for the database server from a configuration file.\cite{wordpress-doc}
The second example is Hibernate, a popular object-relational mapping (ORM) tool, that is written in Java.
Hibernate expects that its login credentials for the database server are provided in an XML configuration file as plain text.\cite{hibernate-doc}

Both applications expect the login credentials to be unencrypted, but storing passwords this way is a major security risk.
However, introducing means of cryptography to an application results in increased development efforts and possibly slower runtime behavior.

\hypothesis{$H_1$}{\hypoOne}
\label{intro-hypo-one}

In order to mitigate the risks of leaking plain text login credentials, they should be encrypted before they are persisted to a storage.
To keep the development effort low for application developers, a software library can abstract the cryptographic operations.

By using third party software libraries an application encounters an increased memory consumption.

\hypothesis{$H_2$}{\hypoTwo}
\label{intro-hypo-two}

Cryptography as a security measure might also bring drawbacks in usability.
In this thesis we are not going to discuss any possible drawbacks regarding usability, but focus solely on the performance analysis.

Cryptographic algorithms and their application have been studied and benchmarked in different contexts.\cite{ocf,freebsdtls,thakur2011aes}
The scope of this thesis is the performance analysis of cryptography applied to application configuration settings.

\section{Elektra}

	\subsection{What is Elektra?}

The \elektra~ project is a configuration management tool, that consists of a library and a set of programs.
The core idea of \elektra~ is to have a centralized hierachical key-value database for configuration settings.
The core of Elektra's source code is written in the C programming language.
\elektra~ is extensible by a plugin system.
Different language bindings offer availability of \elektra~ in other programming languages (for example: Java, Python, and Ruby).
\elektra~ supports common configuration file formats out of the box, for example:\cite{elektra-doc,raab2010thesis}
\begin{enumerate}
\item INI
\item JSON
\item XML
\item Yaml
\end{enumerate}

All technical details, the source code, and the documentation are available online.\footnote{\elektra~ project page: \url{https://www.libelektra.org}}

	\subsection{Elektra And Cryptography}

We choose \elektra~ as our reference project because of its focus on configuration and because of its extensibility.
\elektra~ offers many features, but stores login credentials as plain text originally.
Therefore during the writing of this thesis we developed plugins for \elektra~ that provide transparent encryption and decryption capabilities.

\elektra~ combined with the new plugins solves the problem of transparent encryption and decryption of configuration settings.
This combination is the basis for our experimental evaluation.

\section{Cryptography}

In this section we define what we mean exactly by cryptography, encryption and decryption.

\subsection{Algorithms}

There are many cryptographic algorithms and there are even more implementations.
We can not cover them all, but focus on a typical setting that is viable for most applications.

\subsubsection{Symmetric Cipher}

The Advanced Encryption Standard (AES) is a widely used symmetric block-cipher that is specified in the Federal Information Processing Standards Publication (FIPS) 197.
The FIPS 197 is published by the National Institute of Standards and Technology (NIST).\cite{fips197}
AES supports three key lengths:
\begin{enumerate}
  \item 128 bits
  \item 192 bits
  \item 256 bits
\end{enumerate}

AES operates on data blocks with a size of 128 bits.\cite{fips197,stallings2014}
For the scope of this thesis we choose Cipher Block Chaining (CBC) Mode as the operation mode for AES.
In CBC mode the XOR operation is applied to the plain text and the previous ciphertext block before the actual encryption happens.\cite{bruceschneier1996,stallings2014}

In this thesis we use AES with a key length of 256 bits in CBC mode as a typical symmetric cipher and refer to this combination as \emph{AES-256-CBC}.
We are going to apply AES-256-CBC to single configuration values.
This enables us to protect login credentials within configuration settings.

\subsubsection{Hybrid -- Combining asymmetric and symmetric cryptography}

Asymmetric ciphers tend to be slower than symmetric ciphers.
To mitigate performance problems both asymmetric ciphers and symmetric ciphers are often combined into a public-key cryptographic system.
Such systems utilize asymmetric cryptography to protect a key for a symmetric cipher.
The symmetric cipher protects the actual data.
This way the encrypted payload can be shared among multiple parties without the need to re-encrypt for every recipient.\cite{stallings2014}

The OpenPGP protocol defines such a public-key cryptographic system.
It is specified in the RFC 4880.\cite{rfc4880}
We are going to apply the OpenPGP protocol to encrypt configuration files.
This will protect confidential configuration settings (for example: configuration files that only hold login credentials).

%\subsubsection{Digital Signatures}
%
%The OpenPGP protocol offers the possiblity to attach digital signatures to its messages.\cite{rfc4880}
%With digital signatures the authenticity of a message can be verified.\cite{bruceschneier1996,stallings2014}
%
%We are going to apply digital signatures to configuration files using the OpenPGP protocol.
%Thus we can guarantee that a configuration setting has not been modified or completely replaced by an unknown entity.

\subsection{Providers of Cryptographic Functions}
\label{intro-provider}

We defined which cryptographic algorithms we want to examine.
Next we explain which providers of cryptographic functions we want to use for the examination.

  \subsubsection{GnuPG and libgcrypt}

The GnuPG project is a FLOSS implementation of the OpenPGP protocol.
GnuPG supports:
\begin{enumerate}
\item encryption,
\item decryption,
\item digital signatures, and
\item key management.\cite{gnupg-doc}
\end{enumerate}

The libgcrypt library is a part of the GnuPG project.
The developers of GnuPG encapsulated the low-level implementations of the cryptographic algorithms within libgcrypt.

The source code of GnuPG and libgcrypt is written in the C programming language and is available at the GnuPG project homepage.\footnote{GnuPG project homepage: \url{https://www.gnupg.org}}

  \subsubsection{OpenSSL}

The OpenSSL project offers implementations of the Transport Layer Security (TLS) and the Secure Sockets Layer (SSL) protocols.
It also provides its own implementations of the underlying cryptographic operations, which are accessible via the interfaces of the libcrypto library.

The source code of OpenSSL is written in the C programming language.
It is available at the OpenSSL project homepage.\footnote{OpenSSL project homepage: \url{https://www.openssl.org/}}

  \subsubsection{Botan}

The Botan library is another provider of cryptographic functions.
The source code is written in the C++ programming language and is available at Github.\footnote{Botan's Github page: \url{https://github.com/randombit/botan}}

\section{Research Question}
\label{researchq}

In this thesis we want to examine the following research questions:

\RQ{$RQ_1$}{\rqOne}
\RQ{$RQ_2$}{\rqTwo}
\RQ{$RQ_3$}{\rqThree}

%\section{Perspective}
%
%\todo{TBD}

\chapter{Research Question}

\section{Security VS. System Performance}



\section{Research Question}

\subsection{A Typical Usecase: Encryption Of A Password}

\subsection{Linear Scaling Performance Loss?}

\subsection{Runtime Comparison of Configurations}


\chapter{Experimental Evaluation}

%\section{Methodology}

The focus of this chapter is:
\begin{enumerate}
\item how the benchmark system is set up
\item the exact definition and explanation how the benchmarks work and what is measured
\item the results of the benchmarks
\end{enumerate}

%The runtime of a benchmark will be measured using the system time, which is returned by the \texttt{gettimeofday ()} system function.
%
%The memory usage is examined using the ``Massif'' tool which is part of the \texttt{valgrind} suite.\cite{valgrind}

\section{Benchmark System Setup}
  \subsection{Hardware Setup}

The processor of the benchmark system is an Intel\textregistered~ Core\texttrademark~ i7-4771 CPU clocked at 3.50 GHz with 4 cores and hyperthreading enabled (which means 8 threads are available).
The benchmark system has a total of 16 GB of DDR3 RAM clocked at 1333 MHz available.

The root partition of the operation system is located on a ``Samsung SSD 840'' solid state drive (SSD).
It is installed on a ZFS filesystem.

The \texttt{diskinfo} program gives an idea of how much throughput the SSD can handle:

\begin{lstlisting}[caption={Disk performance on the benchmark system}]
Seek times:
  Full stroke:      250 iter in   0.014774 sec =    0.059 msec
  Half stroke:      250 iter in   0.012662 sec =    0.051 msec
  Quarter stroke:   500 iter in   0.024154 sec =    0.048 msec
  Short forward:    400 iter in   0.018797 sec =    0.047 msec
  Short backward:   400 iter in   0.018726 sec =    0.047 msec
  Seq outer:       2048 iter in   0.075078 sec =    0.037 msec
  Seq inner:       2048 iter in   0.074593 sec =    0.036 msec

Transfer rates:
  outside:       102400 kbytes in   0.221568 sec =   462161 kbytes/sec
  middle:        102400 kbytes in   0.218275 sec =   469133 kbytes/sec
  inside:        102400 kbytes in   0.217949 sec =   469835 kbytes/sec
\end{lstlisting}

All benchmarks are performend on the SSD.

  \subsection{Software Setup}

The operating system FreeBSD version 11.1 with patch level 1 is used to run the benchmarks.
Elektra \todo{version} at the git commit \todo{commit-nr} is installed inside a FreeBSD jail.\footnote{FreeBSD jails
are documented at \url{https://www.freebsd.org/doc/en_US.ISO8859-1/books/handbook/jails.html}.
}
The exact setup of the benchmark jail is explained in appendix \ref{jail-setup} on page \pageref{jail-setup}.

The most important program versions are listed below:

\begin{itemize}
  \item FreeBSD clang version 4.0.0 (tags/RELEASE\_400/final 297347) (based on LLVM 4.0.0)
  \item Botan version 1.10.13\_3
  \item OpenSSL version 1.0.2k-freebsd
  \item libgcrypt version 1.8.0
  \item GnuPG version 2.1.21
\end{itemize}

In the following section the benchmarks are documented and explained in detail.

\section{Benchmark 1 -- TBD}

\section{Interpretation Of The Results}

\chapter{Conclusions}

\section{Résumé}

\section{Further Work}



\backmatter

% Use an optional list of figures.
\listoffigures % Starred version, i.e., \listoffigures*, removes the toc entry.

% Use an optional list of tables.
\listoftables % Starred version, i.e., \listoftables*, removes the toc entry.

% Use an optional list of alogrithms.
\listofalgorithms
\addcontentsline{toc}{chapter}{List of Algorithms}

% Add an index.
\printindex

% Add a glossary.
\printglossaries

% Add a bibliography.
\bibliographystyle{alpha}
\bibliography{literature}

\end{document}