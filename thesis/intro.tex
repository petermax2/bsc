\chapter{Introduction}

\section{Motivation}

Security is a broad field of study and plays an important role in modern day application development.
One aspect of security is the protection of confidential data.
In this thesis we talk about confidentiality, the protection of data against unauthorized access, and integrity, the protection of data against unauthorized change.
Both of these goals are achieved by utilizing cryptography.

While cryptography is useful, does its introduction bring drawbacks as well?
An increase in security usually means a decline in performance and usability.
More protection causes overhead and the key-- and password--management is inconvenient for the user.

We are not going to discuss drawbacks in usability, but focus on the performance issue.
Theoretical analysis of cryptographic algorithms have already been done. \todo{citation needed}
But how do concrete implementations of these algorithms perform in a real application?
In order to find out, we benchmark implementations of cryptographic algorithms within a suitable environment.

In the following section the test environment is introduced.

\section{Environment}

Three providers of cryptographic implementations will be benchmarked:

\begin{enumerate}
\item OpenSSL
\item libgcrypt
\item Botan
\end{enumerate}

They are integrated into the Elektra project, which acts as reference application.

	\subsection{OpenSSL}

The OpenSSL project offers implementations of the Transport Layer Security (TLS) and the Secure Sockets Layer (SSL) protocols.
It also provides its own implementations of the underlying cryptographic operations, which are accessible via the interfaces of the libcrypto library.

The source code of OpenSSL is written in the C programming language.
It is available at the project homepage.\footnote{OpenSSL project homepage: \url{https://www.openssl.org/}}

	\subsection{libgcrypt}

The libgcrypt library is part of the GnuPG project, an open source implementation of the Pretty Good Privacy (PGP) protocol.
The developers of GnuPG encapsulated the low-level implementations of the cryptographic algorithms within libgcrypt.

The source code is also written in C and is available at the GnuPG project homepage.\footnote{GnuPG project homepage: \url{https://www.gnupg.org/}}

	\subsection{Botan}

The Botan library is a free provider of cryptographic functions.

The source code is written in C++ and is available at Github.\footnote{Github project page: \url{https://github.com/randombit/botan}}

	\subsection{Elektra}

The Elektra project is a library for centralized configuration management.
It is extensible by a plugin system that enables the modification of the configuration values in a well-arranged manner.
Elektra also provides a set of tools to directly modify the configuration database.

The core of Elektra's source code is written in C.
The source code is available at the Github project page.\footnote{Elektra project page at Github: \url{https://github.com/ElektraInitiative/libelektra}}

Elektra uses a special terminology.
We adapt to these words and phrases.
Their definitions are provided in the following section.

\section{Terms and Definitions}

\section{Research Question -- Security VS. System Performance}

\section{Related Work}
