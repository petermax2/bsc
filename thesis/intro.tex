\chapter{Introduction}

\section{Motivation}

Security is a broad field of study and plays an important role in modern day application development.
One aspect of security is the protection of confidential data.
In this thesis we talk about confidentiality, the protection of data against unauthorized access, and integrity, the protection of data against unauthorized change.
Both of these goals are achieved by utilizing cryptography.

While cryptography is useful, does its introduction bring drawbacks as well?
An increase in security usually means a decline in performance and usability.
More protection causes overhead and the key-- and password--management is inconvenient for the user.

We are not going to discuss drawbacks in usability, but focus on the performance issue.
Cryptographic algorithms have been studied and benchmarked in different contexts (see \ref{relatedwork} Related Work on page \pageref{relatedwork}).
But how do concrete implementations of these algorithms perform in a real application?
In order to find out, we benchmark implementations of cryptographic algorithms within a suitable test environment.

The following section concretizes the research questions we aim to answer.

\section{Research Question}

The focus of this work is to find out how much the use of cryptography
impacts application performance in terms of runtime and memory
usage.

The research questions this thesis aims to answer are enumerated below:

\begin{enumerate}
\def\labelenumi{\arabic{enumi}.}
\tightlist
\item
  How much more time does the use of cryptography cost (runtime impact)?
\item
  What is the ratio compared to the runtime if cryptography is not used?
\item
  How much more memory is used when using cryptography (memory impact)?
\item
  What is the ratio compared to the memory consumption if cryptography
  is not used?
\end{enumerate}

All of the questions refer to a typical user-space application, that is capable of handling file I/O in different volumes.

\section{Related Work}\label{relatedwork}

Performance analysis of cryptographic operations has been studied by different groups with different interests.

The paper ''DES, AES and Blowfish: Symmetric Key Cryptography Algorithms Simulation Based Performance Analysis`` (\cite{thakur2011aes}) compares implementations of symmetric cryptographic algorithms.
The authors' goal is to find the most performant algorithm.
The paper does not answer how much overhead the introduction of cryptographic methods actually costs, but we will use the results for comparison with our own experiments.

In ''The Design of the OpenBSD Cryptographic Framework`` (\cite{ocf}) developers of OpenBSD describe their kernel interface that abstracts the use of hardware accelerated cryptographic operations.
The authors mainly argue about the performance gain the OCF brings to applications.
But rather than concentrating on a single application the focus of \cite{ocf} is overall system performance in scenarios where multiple applications perform cryptographic operations simultaneously.
Another aspect \cite{ocf} covers is the load-balancing capability OCF has to offer if multiple hardware acceleration cards are available on a system.
\cite{ocf} focuses on the kernel and operating system performance rather than on single applications.
The paper measures speed-up in comparison to cryptographic operations performed in user-space.
Thus the paper does not answer our research questions.

''Improving High-Bandwidth TLS in the FreeBSD kernel`` (\cite{freebsdtls}), another performance study, has been conducted by FreeBSD developers.
They tried to get higher TLS throughput on their high-performance network appliances for Netflix, a video on-demand streaming service.
The focus of \cite{freebsdtls} is the networking aspect considering different network adapters and tuning options in the FreeBSD kernel.
Again the focus is not a single application but rather the improvement of the TLS stack.

Because none of the previous papers answered our research questions, it is reasonable to perform an experimental evaluation to find out the answers.
However, we will refer to \cite{thakur2011aes} and check if our measured overhead correlates with the expected performance loss as described in \cite{thakur2011aes}.

In the following section the test environment is introduced.

\section{Environment}

Three providers of cryptographic implementations will be benchmarked:

\begin{enumerate}
\item OpenSSL
\item libgcrypt
\item Botan
\end{enumerate}

They are integrated into the Elektra project, which acts as reference ''real world`` application.

	\subsection{OpenSSL}

The OpenSSL project offers implementations of the Transport Layer Security (TLS) and the Secure Sockets Layer (SSL) protocols.
It also provides its own implementations of the underlying cryptographic operations, which are accessible via the interfaces of the libcrypto library.

The source code of OpenSSL is written in the C programming language.
It is available at the project homepage.\footnote{OpenSSL project homepage: \url{https://www.openssl.org/}}

	\subsection{libgcrypt}

The libgcrypt library is part of the GnuPG project, an open source implementation of the Pretty Good Privacy (PGP) protocol.
The developers of GnuPG encapsulated the low-level implementations of the cryptographic algorithms within libgcrypt.

The source code is also written in C and is available at the GnuPG project homepage.\footnote{GnuPG project homepage: \url{https://www.gnupg.org/}}

	\subsection{Botan}

The Botan library is a free provider of cryptographic functions.

The source code is written in C++ and is available at Github.\footnote{Github project page: \url{https://github.com/randombit/botan}}

	\subsection{Elektra}

The Elektra project is a library combined with a set of tools for centralized configuration management.
It is extensible by a plugin system that enables the modification of the configuration values in a well-arranged manner.\cite{raab2010thesis}

The plugin framework makes benchmarking easy, which is why we chose Elektra as reference application.
Elektra also provides a set of tools to directly modify the configuration database.

The core of Elektra's source code is written in C.
The source code is available at the Github project page.\footnote{Elektra project page at Github: \url{https://github.com/ElektraInitiative/libelektra}}

\section{Terms and Definitions}

Elektra uses a special terminology, which conflicts with our research topic.
We stick to the following terms in order to keep our statements precise.

\paragraph{Key} is a data structure, that is used by Elektra to combine and hold the following information:
\begin{enumerate}
  \item Name
  \item Metadata
  \item Value
\end{enumerate}
When ``Key'' is used alone we always mean the (Elektra) Key and never a cryptographic key.

\paragraph{KeySet} is an ordered set of (Elektra) Keys.

\paragraph{Cryptographic Key} is a byte sequence used by cryptographic algorithms to enable confidentiality.
We always explicitly refer to cryptography in order to distinguish between (Elektra) Keys and cryptographic keys.

The following chapter describes the changes that have been made to the Elektra project in preparation for benchmarking the impact of cryptography to applications.
