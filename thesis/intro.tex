\chapter{Introduction}

Providing login credentials via application configuration is a common problem in software development.
Think of a web application which connects to a database server or stored tokens for authenticating to third party web-services.
Often these credentials are stored as plain text in a configuration file, leaving them vulnerable to attack.
WordPress is a typical web application with a backend connecting to a database.
The login credentials for the database server are stored as plain text within a configuration file.\cite{wordpress-doc}
Another example is Hibernate, a popular object-relational mapping (ORM) tool written in Java.
Again the login credentials for the database are provided in an XML configuration file as plain text.\cite{hibernate-doc}
These are just two examples but there are many more.

In order to mitigate the risks of leaking plain text login credentials, cryptography should be applied to them.
Passwords and other sensitive information should be stored encrypted.
The decryption process can be designed to be transparent to the application.

While cryptography is useful for our case, its introduction brings drawbacks as well.
Mainly two problems arise:
\begin{enumerate}
\item performance
\item usability
\end{enumerate}

We are not going to discuss drawbacks in usability, but focus on the performance drawbacks.
Cryptographic algorithms and their application have been studied and benchmarked in different contexts.\cite{thakur2011aes,ocf,freebsdtls}
The scope of this thesis is cryptography applied to application configuration.

\section{Elektra}

The Elektra project is a library combined with a set of tools to enable centralized configuration management.
It is extensible by a plugin system that enables the modification of the configuration sets in a well-arranged manner.
The core of Elektra's source code is written in the C programming language.
Different language bindings offer availability in other programming languages (for example: Java, Python, Ruby).
Elektra supports several configuration file formats out of the box.
For example the following file formats can be processed by Elektra:\cite{raab2010thesis,elektra-doc}
\begin{enumerate}
\item INI
\item JSON
\item XML
\item Yaml
\end{enumerate}

All technical details, the source code and the documentation are available online.\footnote{Elektra project page: \url{https://www.libelektra.org}}

We chose Elektra because of its focus on application configuration and its extendebility.
Elektra offers many features, but it would originally store login credentials also as plain text.
During the writing of this thesis we developed plugins for Elektra which provide transparent encryption and decryption.

In the following section we define what performance impacts we want to examine with Elektra.

\section{Research Question}

The focus of this work is to find out how much the use of cryptography
impacts application performance in terms of runtime and memory
usage.

We want to find out if the introduction of cryptography slows down the test application significantly.
Also we want to observe differences in memory usage while running the benchmarks.

The research questions this thesis aims to answer are enumerated below:

\begin{enumerate}
\def\labelenumi{\arabic{enumi}.}
\tightlist
\item
  How much more time does the use of cryptography cost (runtime impact)?
\item
  What is the ratio compared to the runtime if cryptography is not used?
\item
  Are there significant differences between different implementations of cryptographic algorithms?
\item
  How much more memory is used when using cryptography (memory impact)?
\item
  What is the ratio compared to the memory consumption if cryptography
  is not used?
\end{enumerate}

All of the questions refer to a typical user-space application, that is capable of handling file I/O in different volumes.

\section{Environment}

In the following section the test environment is introduced.

Three providers of cryptographic implementations will be benchmarked:

\begin{enumerate}
\item OpenSSL
\item libgcrypt
\item Botan
\end{enumerate}

They are integrated into the reference ''real world`` application.
Elektra is a configuration management library, that also provides applications to interact with.
Its modularity makes it a good choice to act as a reference application.

	\subsection{OpenSSL}

The OpenSSL project offers implementations of the Transport Layer Security (TLS) and the Secure Sockets Layer (SSL) protocols.
It also provides its own implementations of the underlying cryptographic operations, which are accessible via the interfaces of the libcrypto library.

The source code of OpenSSL is written in the C programming language.
It is available at the project homepage.\footnote{OpenSSL project homepage: \url{https://www.openssl.org/}}

	\subsection{libgcrypt}

The libgcrypt library is part of the GnuPG project, an open source implementation of the Pretty Good Privacy (PGP) protocol.
The developers of GnuPG encapsulated the low-level implementations of the cryptographic algorithms within libgcrypt.

The source code is also written in C and is available at the GnuPG project homepage.\footnote{GnuPG project homepage: \url{https://www.gnupg.org/}}

	\subsection{Botan}

The Botan library is another provider of cryptographic functions.

The source code is written in C++ and is available at Github.\footnote{Github project page: \url{https://github.com/randombit/botan}}


\section{Terms and Definitions}

Elektra uses a special terminology, which we need to clarify.
We stick to the following terms in order to keep our statements precise.

\paragraph{Key} is a data structure, that is used by Elektra to combine and hold the following information:
\begin{enumerate}
  \item Name
  \item Metadata
  \item Value
\end{enumerate}
When ``Key'' is used alone we always mean Elektra Key and never a cryptographic key.

\paragraph{Keyset} is an ordered set of keys.

\paragraph{Cryptographic Key} is a byte sequence used by cryptographic algorithms to enable confidentiality.
We always explicitly refer to cryptography in order to distinguish between Elektra keys and cryptographic keys.

\section{Perspective}

The following chapter describes the changes that have been made to the Elektra project in preparation for benchmarking the impact of cryptography to applications.

\todo{TBD}
