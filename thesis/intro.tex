\chapter{Introduction}

Providing login credentials via application configuration is a common problem in software development.
Think of a web application which connects to a database server or stored tokens for authenticating to third party web-services.
These credentials are often stored as plain text in a configuration file, leaving them vulnerable to attack.
WordPress is a typical web application with a backend connecting to a database.
The login credentials for the database server are stored as plain text within a configuration file.\cite{wordpress-doc}
Another example is Hibernate, a popular object-relational mapping (ORM) tool written in Java.
Again the login credentials for the database are provided in an XML configuration file as plain text.\cite{hibernate-doc}
These are just two examples but there are many more.

In order to mitigate the risks of leaking plain text login credentials, cryptography should be applied to them.
Passwords and other sensitive configuration values should be stored encrypted.
The process of encryption and decryption is designed to be transparent to the application.

While cryptography is useful for our case, its introduction brings drawbacks as well.
Mainly two problems arise:
\begin{enumerate}
\item performance
\item usability
\end{enumerate}

We are not going to discuss drawbacks in usability, but focus on the performance drawbacks.
Cryptographic algorithms and their application have been studied and benchmarked in different contexts.\cite{thakur2011aes,ocf,freebsdtls}
The scope of this thesis is cryptography applied to application configuration settings.

\section{Elektra}

The Elektra project is a library combined with a set of tools to enable centralized configuration management.
It is extensible by a plugin system that enables the modification of configuration settings in a well-arranged manner.
The core of Elektra's source code is written in the C programming language.
Different language bindings offer availability of Elektra in other programming languages (for example: Java, Python, Ruby).
Elektra supports common configuration file formats out of the box, for example:\cite{raab2010thesis,elektra-doc}
\begin{enumerate}
\item INI
\item JSON
\item XML
\item Yaml
\end{enumerate}

All technical details, the source code and the documentation are available online.\footnote{Elektra project page: \url{https://www.libelektra.org}}

We choose Elektra as our reference project because of its focus on configuration and because of its extendebility.
Elektra offers many features, but it would still store login credentials as plain text originally.
Therefore during the writing of this thesis we developed plugins for Elektra which provide transparent encryption and decryption capabilities.

\section{Cryptography}

What we mean exactly by encryption and decryption is defined in this section.

\subsection{Algorithms}

There are many cryptographic algorithms out there and there are even more implementations.
We can not (and should not) cover them all but focus on a typical setting that is viable for many applications.

\subsubsection{Symmetric Cipher}

The Advanced Encryption Standard (AES) is a widely used symmetric block-cipher that is specified in the Federal Information Processing Standards Publication (FIPS) 197.
The FIPS 197 is published by the National Institute of Standards and Technology (NIST).\cite{fips197}
AES supports three key lengths:
\begin{enumerate}
  \item 128 bits
  \item 192 bits
  \item 256 bits
\end{enumerate}

AES operates on data blocks with a size of 128 bits.\cite{fips197,stallings2014}
For the scope of this thesis we choose Cipher Block Chaining (CBC) Mode as the operation mode for AES.
In CBC mode the XOR operation is applied to the plain text and the previous ciphertext block before the actual encryption happens.\cite{bruceschneier1996,stallings2014}

In this thesis we use AES with a key length of 256 bits in CBC mode as a typical symmetric cipher and refer to this combination as \emph{AES-256-CBC}.
We are going to apply AES-256-CBC to single configuration values.
This enables us to protect login credentials within configuration settings.

\subsubsection{Hybrid -- Combining asymmetric and symmetric cryptography}

Asymmetric ciphers tend to be slower than symmetric ciphers.
To mitigate performance problems both asymmetric ciphers and symmetric ciphers are often combined into a public-key cryptographic system.
Such systems utilize asymmetric cryptography to protect a key for a symmetric cipher.
The symmetric cipher protects the actual data.
This way the encrypted payload can be shared among multiple parties without the need to re-encrypt for every recipient.\cite{stallings2014} 

The OpenPGP protocol defines such a public-key cryptographic system.
It is specified in the RFC 4880.\cite{rfc4880}
We are going to apply the OpenPGP protocol to encrypt configuration files.
This will protect confidential configuration settings (for example: configuration files that only hold login credentials).

\subsubsection{Digital Signatures}

The OpenPGP protocol offers the possiblity to attach digital signatures to its messages.\cite{rfc4880}
With digital signatures the authenticity of a message can be verified.\cite{bruceschneier1996,stallings2014}

We are going to apply digital signatures to configuration files using the OpenPGP protocol.
Thus we can guarantee that a configuration setting has not been modified by an unknown entity.

\subsection{Providers of Cryptographic Functions}
\label{intro-provider}

We defined which cryptographic algorithms we want to examine.
Next we explain which providers of cryptographic functions we want to use for the examination.

  \subsubsection{GnuPG and libgcrypt}

The GnuPG project is a free open-source implementation of the OpenPGP protocol.
GnuPG supports:\cite{gnupg-doc}
\begin{enumerate}
\item encryption and decryption
\item digital signatures
\item key management
\end{enumerate}

The libgcrypt library is a part of the GnuPG project.
The developers of GnuPG encapsulated the low-level implementations of the cryptographic algorithms within libgcrypt.

The source code of GnuPG and libgcrypt is written in the C programming language and is available at the GnuPG project homepage.\footnote{GnuPG project homepage: \url{https://www.gnupg.org}}

  \subsubsection{OpenSSL}

The OpenSSL project offers implementations of the Transport Layer Security (TLS) and the Secure Sockets Layer (SSL) protocols.
It also provides its own implementations of the underlying cryptographic operations, which are accessible via the interfaces of the libcrypto library.

The source code of OpenSSL is written in the C programming language.
It is available at the OpenSSL project homepage.\footnote{OpenSSL project homepage: \url{https://www.openssl.org/}}

  \subsubsection{Botan}

The Botan library is another provider of cryptographic functions.
The source code is written in the C++ programming language and is available at Github.\footnote{Botan's Github page: \url{https://github.com/randombit/botan}}

\section{Research Question}
\label{researchq}

Considering the scope we defined so far, the following research questions are to be examined:

\begin{enumerate}
  \item Which provider of cryptographic functions is the best fit when comparing the runtime and memory performance of AES-256-CBC?
  \item What runtime and memory overhead is expected on average if AES-256-CBC is applied to configuration values?
  \item What runtime and memory overhead is expected on average if OpenPGP encryption, decryption and signatures are applied to configuration files?
\end{enumerate}

\section{Perspective}

\todo{TBD}
