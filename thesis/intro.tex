\chapter{Introduction}

Storing and providing login credentials is a common problem in software development.
Login credentials in this thesis refer to information that grants access to a system, for example:

\begin{enumerate}
\item passwords, and
\item access tokens (like OAuth tokens).
\end{enumerate}

Login credentials are seen as part of configuration settings of applications.

Applications prefereably store login credentials to related systems in configuration files.
A typical example is a business application that connects to a database system.
The login credentials are often saved as plain text, leaving them vulnerable to attack.
We want to introduce the problem by giving two concrete examples:

\begin{enumerate}
\item WordPress, and
\item Hibernate.
\end{enumerate}

WordPress is a typical web application with a backend connecting to a database.
WordPress reads the login credentials for the database server from a configuration file.\cite{wordpress-doc}
The second example is Hibernate, a popular object-relational mapping (ORM) tool, that is written in Java.
Hibernate expects that its login credentials for the database server are provided in an XML configuration file as plain text.\cite{hibernate-doc}

Both applications expect the login credentials to be unencrypted, but storing passwords this way is a major security risk.
However, introducing means of cryptography to an application results in increased development efforts and possibly slower runtime behavior.

\hypothesis{$H_1$}{\hypoOne}
\label{intro-hypo-1}

In order to mitigate the risks of leaking plain text login credentials, they should be encrypted before they are persisted to a storage.
To keep the development effort low for application developers, a software library can abstract the cryptographic operations.

By using third party software libraries an application encounters an increased memory consumption.

\hypothesis{$H_2$}{\hypoTwo}
\label{intro-hypo-2}

Cryptography as a security measure might also bring drawbacks in usability.
In this thesis we are not going to discuss any possible drawbacks regarding usability, but focus solely on the performance analysis.

Cryptographic algorithms and their application have been studied and benchmarked in different contexts.\cite{ocf,freebsdtls,thakur2011aes}
The scope of this thesis is the performance analysis of cryptography applied to application configuration settings.

\section{Elektra}

	\subsection{What is Elektra?}

The \elektra~ project is a configuration management tool, that consists of a library and a set of programs.
The core idea of \elektra~ is to have a centralized hierachical key-value database for configuration settings.
The core of Elektra's source code is written in the C programming language.
\elektra~ is extensible by a plugin system.
Different language bindings offer availability of \elektra~ in other programming languages (for example: Java, Python, and Ruby).
\elektra~ supports common configuration file formats out of the box, for example:\cite{elektra-doc,raab2010thesis}
\begin{enumerate}
\item INI
\item JSON
\item XML
\item Yaml
\end{enumerate}

All technical details, the source code, and the documentation are available online.\footnote{\elektra~ project page: \url{https://www.libelektra.org}}

	\subsection{Elektra And Cryptography}

We choose \elektra~ as our reference project because of its focus on configuration and because of its extensibility.
\elektra~ offers many features, but stores login credentials as plain text originally.
Therefore during the writing of this thesis we developed plugins for \elektra~ that provide transparent encryption and decryption capabilities.

\elektra~ combined with the new plugins solves the problem of transparent encryption and decryption of configuration settings.
This combination is the basis for our experimental evaluation.

\section{Cryptography}

In this section we define what we mean exactly by cryptography, encryption and decryption.

\subsection{Algorithms}

There are many cryptographic algorithms and there are even more implementations.
We can not cover them all, but focus on a typical setting that is viable for most applications.

\subsubsection{Symmetric Cipher}

The Advanced Encryption Standard (AES) is a widely used symmetric block-cipher that is specified in the Federal Information Processing Standards Publication (FIPS) 197.
The FIPS 197 is published by the National Institute of Standards and Technology (NIST).\cite{fips197}
AES supports three key lengths:
\begin{enumerate}
  \item 128 bits
  \item 192 bits
  \item 256 bits
\end{enumerate}

AES operates on data blocks with a size of 128 bits.\cite{fips197,stallings2014}
For the scope of this thesis we choose Cipher Block Chaining (CBC) Mode as the operation mode for AES.
In CBC mode the XOR operation is applied to the plain text and the previous ciphertext block before the actual encryption happens.\cite{bruceschneier1996,stallings2014}

In this thesis we use AES with a key length of 256 bits in CBC mode as a typical symmetric cipher and refer to this combination as \emph{AES-256-CBC}.
We are going to apply AES-256-CBC to single configuration values.
This enables us to protect login credentials within configuration settings.

\subsubsection{Hybrid -- Combining asymmetric and symmetric cryptography}

Asymmetric ciphers tend to be slower than symmetric ciphers.
To mitigate performance problems both asymmetric ciphers and symmetric ciphers are often combined into a public-key cryptographic system.
Such systems utilize asymmetric cryptography to protect a key for a symmetric cipher.
The symmetric cipher protects the actual data.
This way the encrypted payload can be shared among multiple parties without the need to re-encrypt for every recipient.\cite{stallings2014}

The OpenPGP protocol defines such a public-key cryptographic system.
It is specified in the RFC 4880.\cite{rfc4880}
We are going to apply the OpenPGP protocol to encrypt configuration files.
This will protect confidential configuration settings (for example: configuration files that only hold login credentials).

%\subsubsection{Digital Signatures}
%
%The OpenPGP protocol offers the possiblity to attach digital signatures to its messages.\cite{rfc4880}
%With digital signatures the authenticity of a message can be verified.\cite{bruceschneier1996,stallings2014}
%
%We are going to apply digital signatures to configuration files using the OpenPGP protocol.
%Thus we can guarantee that a configuration setting has not been modified or completely replaced by an unknown entity.

\subsection{Providers of Cryptographic Functions}
\label{intro-provider}

We defined which cryptographic algorithms we want to examine.
Next we explain which providers of cryptographic functions we want to use for the examination.

  \subsubsection{GnuPG and libgcrypt}

The GnuPG project is a FLOSS implementation of the OpenPGP protocol.
GnuPG supports:
\begin{enumerate}
\item encryption,
\item decryption,
\item digital signatures, and
\item key management.\cite{gnupg-doc}
\end{enumerate}

The libgcrypt library is a part of the GnuPG project.
The developers of GnuPG encapsulated the low-level implementations of the cryptographic algorithms within libgcrypt.

The source code of GnuPG and libgcrypt is written in the C programming language and is available at the GnuPG project homepage.\footnote{GnuPG project homepage: \url{https://www.gnupg.org}}

  \subsubsection{OpenSSL}

The OpenSSL project offers implementations of the Transport Layer Security (TLS) and the Secure Sockets Layer (SSL) protocols.
It also provides its own implementations of the underlying cryptographic operations, which are accessible via the interfaces of the libcrypto library.

The source code of OpenSSL is written in the C programming language.
It is available at the OpenSSL project homepage.\footnote{OpenSSL project homepage: \url{https://www.openssl.org/}}

  \subsubsection{Botan}

The Botan library is another provider of cryptographic functions.
The source code is written in the C++ programming language and is available at Github.\footnote{Botan's Github page: \url{https://github.com/randombit/botan}}

\section{Research Question}
\label{researchq}

In this thesis we want to examine the following research questions:

\RQ{$RQ_1$}{\rqOne}
\RQ{$RQ_2$}{\rqTwo}
\RQ{$RQ_3$}{\rqThree}

\section{Perspective}

\todo{TBD}
