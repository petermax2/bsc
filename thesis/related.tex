\chapter{Related and Future Work}

\section{Related Work}\label{relatedwork}

Performance analysis of cryptographic operations has already been studied in different contexts.

	\subsection{Comparing AES Implementations}

The paper ''DES, AES and Blowfish: Symmetric Key Cryptography Algorithms Simulation Based Performance Analysis`` \cite{thakur2011aes} compares implementations of symmetric cryptographic algorithms.
The authors' goal is to find the most performant algorithm.
However, the paper does not answer how much overhead the introduction of cryptographic methods actually costs.\cite{thakur2011aes}

	\subsection{Cryptography in Operating Systems}

In ''The Design of the OpenBSD Cryptographic Framework`` \cite{ocf} developers of OpenBSD describe their kernel interface that abstracts the use of hardware accelerated cryptographic operations.
The authors mainly argue about the performance gain the OCF brings to applications.
But rather than concentrating on a single application the focus of the paper is overall system performance in scenarios where multiple applications perform cryptographic operations simultaneously.
Another aspect the paper covers is the load-balancing capability OCF has to offer, if multiple hardware acceleration cards are available on a system.

The paper's focus is the kernel and operating system performance rather than single application performance.
The paper measures speed-up in comparison to cryptographic operations performed in user-space.\cite{ocf}

''Improving High-Bandwidth TLS in the FreeBSD kernel`` \cite{freebsdtls}, another performance study, has been conducted by FreeBSD developers.
They tried to get higher TLS throughput on their high-performance network appliances for Netflix, a video on-demand streaming service.
The focus of the paper is the networking aspect considering different network adapters and tuning options in the FreeBSD kernel.
Again the focus is not a single application but rather the improvement of the TLS stack in the FreeBSD operating system.\cite{freebsdtls}

	\subsection{Cryptography in Configuration}

Because none of the mentioned papers answered our research questions, it is reasonable to perform an experimental evaluation.

\section{Future Work}

	\subsection{Other criteria for choosing providers of cryptographic functions}

Performance is not the only factor to be taken into account when choosing a provider of cryptographic functions.
Robustness against attacks (for example: side channel attacks) and correctness of the code are two important dimensions, that should also be considered.

Also the usability and the user acceptance play an important role in the decision process.
We did not cover any of those aspects due to the limited context of the thesis.

\todo{TBD}
