\chapter{Related and Future Work}

\section{Related Work}\label{relatedwork}

Performance analysis of cryptographic operations has been studied by different groups with different interests.

The paper ''DES, AES and Blowfish: Symmetric Key Cryptography Algorithms Simulation Based Performance Analysis`` \cite{thakur2011aes} compares implementations of symmetric cryptographic algorithms.
The authors' goal is to find the most performant algorithm.
The paper does not answer how much overhead the introduction of cryptographic methods actually costs, but we will use the results for comparison with our own experiments.

In ''The Design of the OpenBSD Cryptographic Framework`` \cite{ocf} developers of OpenBSD describe their kernel interface that abstracts the use of hardware accelerated cryptographic operations.
The authors mainly argue about the performance gain the OCF brings to applications.
But rather than concentrating on a single application the focus of \cite{ocf} is overall system performance in scenarios where multiple applications perform cryptographic operations simultaneously.
Another aspect \cite{ocf} covers is the load-balancing capability OCF has to offer if multiple hardware acceleration cards are available on a system.
\cite{ocf} focuses on the kernel and operating system performance rather than on single applications.
The paper measures speed-up in comparison to cryptographic operations performed in user-space.
Thus the paper does not answer our research questions.

''Improving High-Bandwidth TLS in the FreeBSD kernel`` \cite{freebsdtls}, another performance study, has been conducted by FreeBSD developers.
They tried to get higher TLS throughput on their high-performance network appliances for Netflix, a video on-demand streaming service.
The focus of \cite{freebsdtls} is the networking aspect considering different network adapters and tuning options in the FreeBSD kernel.
Again the focus is not a single application but rather the improvement of the TLS stack.

Because none of the previous papers answered our research questions, it is reasonable to perform an experimental evaluation to find out the answers.
However, we will refer to \cite{thakur2011aes} and check if our measured overhead correlates with the expected performance loss.

\section{Future Work}

\todo{TBD}