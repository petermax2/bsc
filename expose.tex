% Copyright (C) 2015 Peter Nirschl.
%
% This document is released under the terms of the Creative Commons Attribution-ShareAlike 4.0 International License.
% For further information, please refer to: http://creativecommons.org/licenses/by-sa/4.0/
%

\documentclass[a4paper,12pt]{article}

\usepackage[utf8]{inputenc}
\usepackage[top=3cm, bottom=3cm, left=3cm, right=2cm]{geometry}
\usepackage{fancyhdr}
\usepackage{lastpage}
\usepackage{extramarks}
\usepackage{graphicx}
\usepackage{lipsum}
\usepackage{mathtools}
\usepackage{url}

\linespread{1.1}

% Set up the header and footer
\pagestyle{fancy}
\lhead{}
\chead{Exposé}
\rhead{}
\lfoot{}
\cfoot{\thepage}
\rfoot{}
\renewcommand\headrulewidth{0.4pt}
%\renewcommand\footrulewidth{0.4pt}

% Setup paragraph style
\setlength\parindent{0pt} % Removes all indentation from paragraphs
\setlength{\parskip}{0.2cm}

%----------------------------------------------------------------------------------------
%	USER DEFINED COMMANDS
%----------------------------------------------------------------------------------------
\newcommand{\libelektra}{\texttt{libelektra}~}
\newcommand{\libgcrypt}{\texttt{libgcrypt}~}


%----------------------------------------------------------------------------------------
%	TITLE PAGE
%----------------------------------------------------------------------------------------
\title{Exposé}
\author{Peter Nirschl (1025647)}
\date{Version 0.1}
%----------------------------------------------------------------------------------------

\begin{document}

\maketitle
\begin{abstract}
This document is the basic orientation guideline for my Bachelor thesis.
It defines the research goals and lists the related literature as well as the planned timeline.
\end{abstract}
%\setcounter{tocdepth}{1} % defines whether or not subsections should not be listed in the TOC
%\tableofcontents
%\newpage

\vfill

\section*{Copyright Notice}

Copyright \copyright~ 2015 Peter Nirschl.

This work is released under the terms of the \textbf{Creative Commons Attribution-ShareAlike 4.0 International License}.
For further information, please refer to:

\url{http://creativecommons.org/licenses/by-sa/4.0/}

\newpage

%----------------------------------------------------------------------------------------
%	TITLE
%----------------------------------------------------------------------------------------
\section{Title}

Implementation and Performance Analysis of Cryptographic Methods for Elektra


%----------------------------------------------------------------------------------------
%	TASK DESCRIPTION
%----------------------------------------------------------------------------------------
\section{Task Description}

Security has become a crucial aspect of modern day applications, especially when it comes to libraries.
\libelektra (or simply Elektra) is essentially a hierarchical key-value database that is intended to store configuration data. It features extensibility by providing a plugin system. The main reason why \libelektra should provide security measures is that application configurations often contain sensitive data (e.g. login credentials).

The goal is to enhance \libelektra to provide \textbf{confidentiality} for sensitive values.
Those values should be protected against unauthorized usage by means of symmetric encryption.
An existing implementation of the Advanced Encryption Standard (AES) will be used for encryption.

Another security enhancement to \libelektra is to provide \textbf{integrity} for configuration files.
Therefore an RSA-based signature algorithm will be used to detect unauthorized changes between the time the file has been persisted and reloaded again.

However, security comes at a cost, namely: comfort and performance. The (possible) influences of security measures on the user experience are not covered in this thesis. The main question will be if the introduction of cryptographic methods leads to a significant decline in performance.

The following section describes the research questions, which arise from the task description.


%----------------------------------------------------------------------------------------
%	RESEARCH QUESTION
%----------------------------------------------------------------------------------------
\section{Research Questions}

Let $t_e(x)$ denote the time the encryption of a value $x$ takes, and let $t_d(x)$ denote the time the decryption of a value $x$ takes.
The research questions examine the possible performance impact of cryptographic operations on Elektra:

\begin{enumerate}
\item How long does it take to encrypt and decrypt a typical password $p$, i.e. what are the mean values of $t_e(p)$ and $t_d(p)$? //TODO add literature ref about typical password strength
  
\item How big is the ratio of the time needed to encrypt a value $x$ in comparison to the total time of the persisting operation $T_p(x)$ for the value $x$, i.e. what is the mean value of $t_e(x)/T_p(x)$?
  
\item How big is the ratio of the time needed to decrypt a value $x$ in comparison to the total time of the loading operation $T_l(x)$ for the value $x$, i.e. what is the mean value of $t_d(x)/T_l(x)$?

\item How much impact has the length of the value $|x|$ on $t_d(x)$ and $t_e(x)$, i.e. what is the (statistical) asymptotic behaviour of $t_e(x)$ and $t_d(x)$?

\item How much time is needed for signing and verifying Elektra database files of different sizes?
  
\end{enumerate}

The next section provides details about how these questions will be examined.


%----------------------------------------------------------------------------------------
%	METHODOLOGY
%----------------------------------------------------------------------------------------
\section{Methodology}

So far the research question as well as the task description have been discussed.
This section provides detailed information about how the research questions can be answered step by step.

The \libelektra project is being used as a test environment, as mentioned in the task description.
The source code of the \libelektra project is available at GitHub:

\url{https://github.com/ElektraInitiative/libelektra/}

The cryptographic operations will be provided by \libgcrypt, an open source cryptographic API, which is FIPS compliant and can thus be considered as ``sufficiently secure''.
This project will act as the real-world implementation and will be tested for performance in order to examine the research question.
The \libgcrypt project is hosted by the Free Software Foundation (FSF) and can be accessed at:

\url{http://www.gnu.org/software/libgcrypt/}

The first step of the project is the integration of security concepts in \libelektra. Confidentiality will be increased by providing encrypted values. In order to distinguish encrypted values from plain text, the key meta-informations will be extended. Encrypted valus will be stored in Base64 encoding (RFC 4848, see \url{https://tools.ietf.org/html/rfc4648}). The Advanced Encryption Standard (AES) will be used as symmetric cipher.

Once the symmetric encryption operations are integrated, test cases will be created that enable performance measurements. To see how much time is spent in the encryption and decryption functions the tool \texttt{callgrind} will be used. See:

\url{http://www.valgrind.org/docs/manual/cl-manual.html}

Finally the integrity of the key database files will be assured by storing the signed SHA-512 hash value of the file content. RSA will be used for signing the hash values. If handling the RSA keypair for the signature procedure should decrease usability, integrity can be provided by storing a hashed message authentication code (HMAC) alternatively. \libgcrypt supports all of these cryptographic operations, so changing the method for assuring integrity should not be much effort.

Again test cases will be used to measure the overall performance of the signing and verfifying operations.

Now the strategy for examining the research questions is clear. The following section will explain the planned timeline.

%----------------------------------------------------------------------------------------
%	TIMELINE
%----------------------------------------------------------------------------------------
\section{Timeline}

The thesis is separated into several milestones. The following table is a timeline, that shows when each of the deliverables should be ready.

\begin{table}[h]
\begin{tabular}{ll}
\textbf{Date} & \textbf{Deliverable / Artifact}                                           \\ \hline
2015/04/30    & Exposé                                                                    \\ \hline
2015/05/31    & Concept (software design) for symmetric cryptographic implementation      \\
              & Thesis: Description of the methodology                                    \\ \hline
2015/06/30    & Completed implementation of symmetric cryptography                        \\ \hline
2015/07/31    & Concept (software design)  for the asymmetric cryptography implementation \\
              & Thesis: completed sections about symmetric cryptography                   \\ \hline
2015/08/31    & Completed implementation of asymmetric cryptogryphy                       \\
              & Thesis: completed sections about asymmetric cryptography                  \\
              & Thesis: introduction and conclusion (draft)                               \\ \hline
2015/09/31    & Review \& final thesis                                                   
\end{tabular}
\end{table}

%----------------------------------------------------------------------------------------
%	RELATED WORK
%----------------------------------------------------------------------------------------
\section{Related Work}

\begingroup
\renewcommand{\section}[2]{}
\renewcommand{\refname}{}
\nocite{*}
\bibliography{literature}
\bibliographystyle{alpha}
\endgroup


\end{document}
