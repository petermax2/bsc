% Copyright (C) 2015 Peter Nirschl.
%
% This document is released under the terms of the Creative Commons Attribution-ShareAlike 4.0 International License.
% For further information, please refer to: http://creativecommons.org/licenses/by-sa/4.0/
%

\documentclass[a4paper,12pt]{article}

\usepackage[utf8]{inputenc}
\usepackage[top=3cm, bottom=3cm, left=3cm, right=2cm]{geometry}
\usepackage{fancyhdr}
\usepackage{lastpage}
\usepackage{extramarks}
\usepackage{graphicx}
\usepackage{lipsum}
\usepackage{mathtools}
\usepackage{url}

\linespread{1.1}

% Set up the header and footer
\pagestyle{fancy}
\lhead{Exposé -- Increasing confidentiality and integrity in a hierarchical key-value database}
\chead{}
\rhead{}
\lfoot{}
\cfoot{\thepage}
\rfoot{}
\renewcommand\headrulewidth{0.4pt}
%\renewcommand\footrulewidth{0.4pt}

% Setup paragraph style
\setlength\parindent{0pt} % Removes all indentation from paragraphs
\setlength{\parskip}{0.2cm}

%----------------------------------------------------------------------------------------
%	USER DEFINED COMMANDS
%----------------------------------------------------------------------------------------
\newcommand{\libelektra}{\texttt{libelektra}~}
\newcommand{\libgcrypt}{\texttt{libgcrypt}~}


%----------------------------------------------------------------------------------------
%	TITLE PAGE
%----------------------------------------------------------------------------------------
\title{Exposé}
\author{Peter Nirschl (1025647)}
\date{Version 0.1}
%----------------------------------------------------------------------------------------

\begin{document}

\maketitle
\begin{abstract}
This document is the basic orientation guideline for my Bachelor thesis.
It defines the research goals and lists the related literature as well as the planned timeline.
\end{abstract}
%\setcounter{tocdepth}{1} % defines whether or not subsections should not be listed in the TOC
%\tableofcontents
%\newpage

\vfill

\section*{Copyright Notice}

Copyright \copyright~ 2015 Peter Nirschl.

This work is released under the terms of the \textbf{Creative Commons Attribution-ShareAlike 4.0 International License}.
For further information, please refer to:

\url{http://creativecommons.org/licenses/by-sa/4.0/}

\newpage

%----------------------------------------------------------------------------------------
%	TITLE
%----------------------------------------------------------------------------------------
\section{Title}

Increasing confidentiality and integrity in a hierarchical key-value database


%----------------------------------------------------------------------------------------
%	TASK DESCRIPTION
%----------------------------------------------------------------------------------------
\section{Task Description}

Security has become a crucial aspect of modern day applications, especially when it comes to libraries.
\libelektra is essentially a hierarchical key-value database that features extensibility by providing a plugin system.
There are two main reasons why such a project should provide security measures:

\begin{enumerate}
\item The library represents a single point of failure to all using applications, as the code is being shared.
\item \libelektra is dealing with application configurations, which often contain sensitive data like login credentials or private keys.
\end{enumerate}

The goal is to enhance \libelektra to provide \textbf{confidentiality} for stored username- and password values.
Those values will be protected against unauthorized usage by means of symmetric encryption.
An existing implementation of the Advanced Encryption Standard (AES) will be used for encryption.

Another security enhancement to \libelektra is to provide \textbf{integrity} for configuration files.
Therefore an RSA-based signature algorithm will be used to detect unauthorized changes between the time the file has been persisted and reloaded again.

The following section describes the research questions, which arise from the task description.


%----------------------------------------------------------------------------------------
%	RESEARCH QUESTION
%----------------------------------------------------------------------------------------
\section{Research Question}

The research question examines the practical applications of cryptography in a cross-platform library environment and the possible increase in data security.
It consists of the following two parts:

\begin{enumerate}
\item Does a real-world implementation of a symmetric cryptographic algorithm offer confidentiality (i.e. protection against unauthorized access) for small data units?
Small data refers to values, that possibly fit into a single encryption block (e.g. passwords).
\item Can a real-world implementation of an asymmetric digital signature algorithm offer integrity (i.e. protection against unauthorized modification) for configuration files, that are persisted on a storage device (e.g. a hard drive)?
\end{enumerate}

The next section provides details about how these questions will be examined.


%----------------------------------------------------------------------------------------
%	METHODOLOGY
%----------------------------------------------------------------------------------------
\section{Methodology}

So far the research question as well as the task description have been discussed.
This section provides detailed information about how the research questions can be answered step by step.

The \libelektra project is being used as a test environment, as mentioned in the task description.
The source code of the \libelektra project is available at GitHub:

\url{https://github.com/ElektraInitiative/libelektra/}

The cryptographic operations will be provided by \libgcrypt, an open source cryptographic API, which is FIPS compliant and can thus be considered as ``sufficiently secure''.
This project will act as the real-world implementation and will be tested against the security threats in order to examine the research question.
The \libgcrypt project is hosted by the Free Software Foundation (FSF) and can be accessed at:

\url{http://www.gnu.org/software/libgcrypt/}

The first step is the integration of the security concepts into \libelektra. Confidentiality will be increased by providing encrypted values. In order to distinguish encrypted values and plain text, the key meta-informations (source file \texttt{keymeta.c}) will be extended. Encrypted valus will be persisted in Base64 encoding (RFC 4848, see \url{https://tools.ietf.org/html/rfc4648}). The Advanced Encryption Standard (AES) will be used as symmetric cipher.

Once the symmetric encryption operations are integrated, test databases will be created. These databases are used to test the algorithm against known vulnerabilities. Security tests are either unit tests, that are integrated into the existing test suite, or stand-alone applications, depending on the complexity of the test setup and the required environment. The outcomes of these tests are evaluated and interpreted to answer part one of the research question.

Finally the integrity of the key database files will be assured by storing the signed SHA-512 hash value of the file content. RSA will be used for signing the hash values. If the handling of the RSA keypair for the signature procedure should decrease usability, integrity can be provided by storing a hashed message authentication code (HMAC). \libgcrypt supports all of these cryptographic operations, so changing the method for assuring integrity should not be much effort.

Again test cases will be used to attack known vulnerabilities of the implementations. Part two of the research question will be answered by evaluating and interpreting the outcomes of these tests.



%----------------------------------------------------------------------------------------
%	TIMING
%----------------------------------------------------------------------------------------
\section{Timing}

\begin{table}[h]
\begin{tabular}{ll}
\textbf{Date} & \textbf{Deliverable / Artifact} \\
2015/04/30 & Concept (software design) for symmetric cryptographic implementation \\
2015/05/31 & Thesis: Description of the methodology, first draft for test cases \\
2015/06/30 & Completed work on symmetric cryptography (part 1 of the research question) \\
2015/07/31 & Concept and description of the asymmetric cryptography implementation \\
2015/08/31 & Completed work on asymmetric cryptogryphy (part 2 of the research question) \\
~          & conclusion \\ 
2015/09/31 & Review \& finalization
\end{tabular}
\end{table}

%----------------------------------------------------------------------------------------
%	RELATED WORK
%----------------------------------------------------------------------------------------
\section{Related Work}

\begingroup
\renewcommand{\section}[2]{}
\renewcommand{\refname}{}
\nocite{*}
\bibliography{literature}
\bibliographystyle{alpha}
\endgroup


\end{document}
