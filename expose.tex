% Copyright (C) 2015 Peter Nirschl.
%
% This document is released under the terms of the Creative Commons Attribution-ShareAlike 4.0 International License.
% For further information, please refer to: http://creativecommons.org/licenses/by-sa/4.0/
%

\documentclass[a4paper]{article}

\usepackage[utf8]{inputenc}
\usepackage{fancyhdr}
\usepackage{lastpage}
\usepackage{extramarks}
\usepackage{graphicx}
\usepackage{lipsum}
\usepackage{mathtools}
\usepackage{url}

% Margins
%\topmargin=0.5cm
%\evensidemargin=1.0cm
%\oddsidemargin=1.0cm
%\textwidth=6.5in
%\textheight=9.0in
%\headsep=0.25in 

\linespread{1.1}

% Set up the header and footer
\pagestyle{fancy}
\lhead{}
\chead{Exposé}
\rhead{}
\lfoot{Peter Nirschl (1025647)}
\cfoot{}
\rfoot{Page\ \thepage\ of\ \pageref{LastPage}}
\renewcommand\headrulewidth{0.4pt}
\renewcommand\footrulewidth{0.4pt}

% Setup paragraph style
\setlength\parindent{0pt} % Removes all indentation from paragraphs
\setlength{\parskip}{0.2cm}

%----------------------------------------------------------------------------------------
%	USER DEFINED COMMANDS
%----------------------------------------------------------------------------------------
\newcommand{\libelektra}{\texttt{libelektra}~}


%----------------------------------------------------------------------------------------
%	TITLE PAGE
%----------------------------------------------------------------------------------------
\title{Exposé}
\author{Peter Nirschl (1025647)}
\date{\today} % Insert date here if you want it to appear below your name
%----------------------------------------------------------------------------------------

\begin{document}

\maketitle
\begin{abstract}
This document is the basic orientation guideline for my Bachelor thesis.
It defines the research goals and lists the related literature as well as the planned timeline.
\end{abstract}
\setcounter{tocdepth}{1} % defines whether or not subsections should not be listed in the TOC
\tableofcontents
\newpage

\section*{Copyright Notice}

Copyright \copyright~ 2015 Peter Nirschl.

This work is released under the terms of the \textbf{Creative Commons Attribution-ShareAlike 4.0 International License}.
For further information, please refer to:

\url{http://creativecommons.org/licenses/by-sa/4.0/}

\newpage

%----------------------------------------------------------------------------------------
%	TITLE
%----------------------------------------------------------------------------------------
\section{Title}

Increasing confidentiality and integritiy in a hierarchical key-value database


%----------------------------------------------------------------------------------------
%	RESEARCH QUESTION
%----------------------------------------------------------------------------------------
\section{Research Question}

The research question in this thesis examines the practical applications of cryptography in a cross-platform library environment and the possible increase in data security.
It consists of the following two parts.

\begin{enumerate}
\item Does a real-world Advanced Encryption Standard (AES) implementation offer confidentiality (i.e. protection against unauthorized access) for small data units?
Small data refers to values, that possibly fit into a single encryption block (e.g. passwords).
      
\item Can a real-world implementation of an RSA-based digital signature algorithm offer integrity (i.e. protection against unauthorized modification) for configuration files, that are persisted on a storage device (e.g. a hard drive)?

\end{enumerate}

The following sections provide further details about how these questions will be examined.


%----------------------------------------------------------------------------------------
%	TASK DESCRIPTION
%----------------------------------------------------------------------------------------
\section{Task Description}

Security has become a crucial aspect of modern day applications.
Especially when it comes to libraries, which support different applications on various operating systems.
\libelektra\footnote{\url{https://github.com/ElektraInitiative/libelektra/}} is essentially a hierachical key-value database that features extensibility by providing a plugin system.
There are two main reasons why such a project should provide security measures.
The first reason is that the library represents a single point of failure to all using applications, as the code is being shared.
If the library is vulnerable, all the applications, which use its features, are vulnerable too.
The second reason is that \libelektra is dealing with application configurations, which often contain sensitive data like login credentials or private keys.

The idea of this thesis is to enhance \libelektra to provide \textbf{confidentiality} for stored username- and password values.
Those values will be protected against unauthorized usage by means of symmetric encryption.
An existing implementation of the Advanced Encryption Standard (AES) will be used for encryption.

Another security enhancement to \libelektra is to provide \textbf{integrity} for configuration files.
Therefor an RSA-based signature algorithm will be used to detect unauthorized changes between the time the file has been persisted and reloaded again.

The following section describes how these tasks can be implemented and how the research questions can be answered.


%----------------------------------------------------------------------------------------
%	METHODOLOGY
%----------------------------------------------------------------------------------------
\section{Methodology}

So far the research question as well as the task description have been discussed in a general way.
This section provides detailed information about how the research questions can be answered step by step.

To examine the research question the \libelektra project is being used, as mentioned in the task description.
The source code of the project is available at GitHub (\url{https://github.com/ElektraInitiative/libelektra/}).



%----------------------------------------------------------------------------------------
%	TIMING
%----------------------------------------------------------------------------------------
\section{Timing}
% set milestones (including deliverables) and set delivery dates
TBD
  

%----------------------------------------------------------------------------------------
%	RELATED WORK
%----------------------------------------------------------------------------------------
\section{Related Work}

\begingroup
\renewcommand{\section}[2]{}
\renewcommand{\refname}{}
\nocite{*}
\bibliography{literature}
\bibliographystyle{alpha}
\endgroup


\end{document}
